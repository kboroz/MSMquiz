\documentclass{beamer}

\usepackage{graphicx}
\usepackage[latin1]{inputenc}
\usepackage[T1]{fontenc}
\usepackage[english]{babel}
\usepackage{listings}
\usepackage{xcolor}
\usepackage{eso-pic}
\usepackage{mathrsfs}
\usepackage{url}
\usepackage{amssymb}
\usepackage{amsmath}
\usepackage{multirow}
\usepackage{hyperref}
\usepackage{booktabs}
\usepackage{bbm}
\usepackage{cooltooltips}
\usepackage{colordef}
\usepackage{beamerdefs}
\usepackage{lvblisting}

\pgfdeclareimage[height=3.5cm]{logobig}{hucaselogo}
\pgfdeclareimage[height=0.7cm]{logosmall}{Figures/LOB_Logo}

\renewcommand{\titlescale}{1.0}
\renewcommand{\titlescale}{1.0}
\renewcommand{\leftcol}{0.6}

\title[Selected Topics of Mathematical Statistics]{Title}
\authora{Bj�rn Bokelmann}
\authorb{}
\authorc{}

\def\linka{http://lvb.wiwi.hu-berlin.de}
\def\linkb{http://case.hu-berlin.de}
\def\linkc{}

\institute{Ladislaus von Bortkiewicz Chair of Statistics \\
C.A.S.E. -- Center for Applied Statistics\\
and Economics\\
Humboldt--Universit�t zu Berlin \\}

\hypersetup{pdfpagemode=FullScreen}

\begin{document}


%%%%%%%%%%%%%%%%%%%%%%%%%%%%%%%%%%%%%%%%
\section{Further Discussion of Convergence in Distribution}
%%%%%%%%%%%%%%%%%%%%%%%%%%%%%%%%%%%%%%%%
\frame{
\color{brown}
Quiz 11
\color{black}
\\
by Bj�rn Bokelmann
}
%%%%%%%%%%%%%%%%%%%%%%%%%%%%%%%%%%%%%%%%
\frame{
%Exercise
\color{brown}
Given that $X_{n}=AN(n,2n)$, why does $\frac{n-1}{n}X_{n}=AN(n,2n)$ hold and why is $\frac{n^{1/2}-1}{n^{1/2}}X_{n}\not\sim AN(n,2n)$? \newline

\color{black}
%First Part
First we define $a_{n}=\frac{n-1}{n}$, $b_{n}=0$, $\mu_{n}=n$, $\sigma_{n}=\sqrt{2n}$. \\
Then $X_{n}=AN(\mu_{n},\sigma_{n})$. Further it holds 
\begin{align*}
a_{n}=\frac{n-1}{n}=1+\frac{1}{n} \stackrel{n\rightarrow \infty}{\rightarrow} 1
\end{align*}
and 
\begin{align*}
\frac{\mu_{n}(a_{n}-1)+b_{n}}{\sigma_{n}}=\frac{n(\frac{n-1}{n}-1)}{\sqrt{2n}}=\frac{-1}{\sqrt{2n}}\stackrel{n\rightarrow \infty}{\rightarrow} 0 \ .
\end{align*}
After Lemma 23 $\frac{n-1}{n}X_{n}$ is $AN(\mu_{n},\sigma_{n}).$ 
}

\frame{
%Exercise
\color{brown}
Given that $X_{n}=AN(n,2n)$, why does $\frac{n-1}{n}X_{n}=AN(n,2n)$ hold and why is $\frac{n^{1/2}-1}{n^{1/2}}X_{n}\not\sim AN(n,2n)$ ? \newline




\color{black}
%Second Part
Again we define $a_{n}=\frac{n^{1/2}-1}{n^{1/2}}$, $b_{n}=0$, $\mu_{n}=n$, $\sigma_{n}=\sqrt{2n}$. \\ 
Then it holds
\begin{align*}
\frac{\mu_{n}(a_{n}-1)+b_{n}}{\sigma_{n}}=\frac{n(\frac{n^{1/2}-1}{n^{1/2}}-1)}{\sqrt{2n}}=\frac{n-n^{1/2}-n}{\sqrt{2n}}=-\frac{1}{\sqrt{2}}\stackrel{n\rightarrow \infty}{\not\rightarrow}0
\end{align*}
Again after Lemma 23 we can conclude that $a_{n}X_{n}\not\sim AN(n,2n)$.
}

\frame{

An example for a sequence $X_{n}\sim AN(n,2n)$ is \\
$X_{n}=3 \underset{i=1}{\overset{n}{\sum}} Y_{n}$ with $Y_{n}\sim Ber(\frac{1}{3})$. 

\begin{proof}
Because $E[3Y_{i}]=3\cdot \frac{1}{3}=1$ and $Var[3Y_{i}]=9\cdot \frac{1}{3}(1-\frac{1}{3})=2$ it holds after the Central Limit Theorem 
\begin{align*}
\frac{\underset{i=1}{\overset{n}{\sum}} 3Y_{n}-n}{\sqrt{2n}}\stackrel{\mathcal{L}}{\rightarrow}N(0,1)
\end{align*}
Thus $X_{n}=\underset{i=1}{\overset{n}{\sum}} 3Y_{n}\sim AN(n,2n)$.
\end{proof}

}








\end{document}
