%----------------------------------------------------------------------------------------
%	PACKAGES AND OTHER DOCUMENT CONFIGURATIONS
%----------------------------------------------------------------------------------------

\documentclass[paper=a4, fontsize=11pt]{scrartcl} % A4 paper and 11pt font size

\usepackage[T1]{fontenc} % Use 8-bit encoding that has 256 glyphs
\usepackage{fourier} % Use the Adobe Utopia font for the document - comment this line to return to the LaTeX default
\usepackage[english]{babel} % English language/hyphenation
\usepackage{amsmath,amsfonts,amsthm} % Math packages

\usepackage{sectsty} % Allows customizing section commands
\allsectionsfont{\centering \normalfont\scshape} % Make all sections centered, the default font and small caps

\usepackage{fancyhdr} % Custom headers and footers
\pagestyle{fancyplain} % Makes all pages in the document conform to the custom headers and footers
\fancyhead{} % No page header - if you want one, create it in the same way as the footers below
\fancyfoot[L]{} % Empty left footer
\fancyfoot[C]{} % Empty center footer
\fancyfoot[R]{\thepage} % Page numbering for right footer
\renewcommand{\headrulewidth}{0pt} % Remove header underlines
\renewcommand{\footrulewidth}{0pt} % Remove footer underlines
\setlength{\headheight}{13.6pt} % Customize the height of the header

\numberwithin{equation}{section} % Number equations within sections (i.e. 1.1, 1.2, 2.1, 2.2 instead of 1, 2, 3, 4)
\numberwithin{figure}{section} % Number figures within sections (i.e. 1.1, 1.2, 2.1, 2.2 instead of 1, 2, 3, 4)
\numberwithin{table}{section} % Number tables within sections (i.e. 1.1, 1.2, 2.1, 2.2 instead of 1, 2, 3, 4)

\setlength\parindent{0pt} % Removes all indentation from paragraphs - comment this line for an assignment with lots of text

%----------------------------------------------------------------------------------------
%	TITLE SECTION
%----------------------------------------------------------------------------------------

\newcommand{\horrule}[1]{\rule{\linewidth}{#1}} % Create horizontal rule command with 1 argument of height

\title{	
\normalfont \normalsize 
\textsc{Humboldt University Berlin} \\ [25pt] 
\horrule{0.5pt} \\[0.4cm] 
\huge Quiz 1 \\ 
\horrule{2pt} \\[0.5cm] 
}

\author{Malte Esders} 

\date{\normalsize\today} 

\begin{document}

\maketitle % Print the title

\section{Problem Description}
Quiz 1: Relate Interquartile Range (IQR) to Standard Deviation (SD) under some distributions.
\newpage

\section{Solution}

\subsection{Uniform Distribution}
Probability density function:
\begin{align} 
\begin{split}
	f_X(x) = \frac{1}{b-a}
\end{split}					
\end{align}

Cumulative density function:
\begin{align} 
\begin{split}
	F_X(x) = \frac{x-a}{b-a}
\end{split}					
\end{align}

Quantile function:
\begin{equation}
\begin{aligned} 
	F_X^{-1}(p) = a + p(b-a)  \\
\end{aligned}
\end{equation}

The quantiles:
\begin{equation}
\begin{aligned} 
	\xi_{\frac{1}{4}} = F^{-1}(\tfrac{1}{4}) = a + \frac{1}{4}(b-a) \\
	\xi_{\frac{3}{4}} = F^{-1}(\tfrac{3}{4}) = a + \frac{3}{4}(b-a)
\end{aligned}
\end{equation}

So, the Interquartile Range is
\begin{equation}
	IQR = \frac{1}{2}(b-a)
\end{equation}

And the Standard Deviation of the uniform distribution is known to be $\frac{(b-a)}{\sqrt{12}}$, so the ratio of IQR to SD is

\begin{equation}
	\frac{IQR}{SD} = \frac{\frac{1}{2}}{\frac{1}{\sqrt{12}}} = \sqrt{3}
\end{equation}


\subsection{Exponential Distribution}

Probability density function:
\begin{align} 
\begin{split}
	f_X(x) = \lambda e^{-\lambda x}
\end{split}					
\end{align}

Cumulative density function:
\begin{align} 
\begin{split}
	F_X(x) = 1-e^{-\lambda x}
\end{split}					
\end{align}

Deriving the quantile function:
\begin{equation}
\begin{aligned} 
	F_X^{-1}(p) &= x_p  \\
	p &= 1-e^{-\lambda x_p} \\
	e^{-\lambda x_p} &= 1-p \\
	x_p &= -\frac{\log(1-p)}{\lambda} \\
\end{aligned}
\end{equation}

So, $F_X^{-1}(p)$ is: 
\begin{align} 
\begin{split}
	F_X^{-1}(p) = -\frac{\log(1-p)}{\lambda}
\end{split}					
\end{align}

\begin{equation}
\begin{aligned} 
	\xi_{\frac{1}{4}} = F^{-1}(\tfrac{1}{4}) = -\frac{\log(1-\frac{1}{4})}{\lambda} = \frac{\log{\frac{4}{3}}}{\lambda}\\
	\xi_{\frac{3}{4}} = F^{-1}(\tfrac{3}{4}) = -\frac{\log(1-\frac{3}{4})}{\lambda} = \frac{\log{4}}{\lambda}
\end{aligned}
\end{equation}

So, the Interquartile Range is
\begin{equation}
	IQR = \frac{\log{4} - \log{\frac{4}{3}}}{\lambda} = \frac{\log{3}}{\lambda}
\end{equation}

And the Standard Deviation of the exponential distribution is known to be $\frac{1}{\lambda^2}$, so the ratio of IQR to SD is

\begin{equation}
	\frac{IQR}{SD} = \lambda \log{3} \approx \lambda * 1.098
\end{equation}


\subsection{Standard Normal Distribution}
For the Normal Distribution, the cumulative and quantile distributions can't be expressed in elementary functions, therefore one has to use numerical approximations.\\
The first and third quantiles are

\begin{equation}
\begin{aligned} 
	\xi_{\frac{1}{4}} = \Phi^{-1}(\tfrac{1}{4}) &= -0.68 \\
	\xi_{\frac{3}{4}} = \Phi^{-1}(\tfrac{3}{4}) &= 0.68
\end{aligned}
\end{equation}
\\
So the interquartile range is $IQR = 0.68-(-0.68) = 1.36$. Since the standard deviation of the standard normal distribution is 1, the ratio is

\begin{equation}
	\frac{IQR}{SD} = \frac{1.36}{1} = 1.36
\end{equation}

\end{document}

%%%%%%%%%%%%%%%%%%%%%%%%%%%%%%%%%%%%%%%%%
% Short Sectioned Assignment
% LaTeX Template
% Version 1.0 (5/5/12)
%
% This template has been downloaded from:
% http://www.LaTeXTemplates.com
%
% Original author:
% Frits Wenneker (http://www.howtotex.com)
%
% License:
% CC BY-NC-SA 3.0 (http://creativecommons.org/licenses/by-nc-sa/3.0/)
%
%%%%%%%%%%%%%%%%%%%%%%%%%%%%%%%%%%%%%%%%%
