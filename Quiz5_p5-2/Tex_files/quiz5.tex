% Type of the document
\documentclass{beamer}

% elementary packages:
\usepackage{graphicx}
\usepackage[latin1]{inputenc}
\usepackage[T1]{fontenc}
\usepackage[english]{babel}
\usepackage{listings}
\usepackage{xcolor}
\usepackage{eso-pic}
\usepackage{mathrsfs}
\usepackage{url}
\usepackage{amssymb}
\usepackage{amsmath}
\usepackage{multirow}
\usepackage{hyperref}
\usepackage{booktabs}
\usepackage{dsfont}
% additional packages
\usepackage{bbm}

% packages supplied with ise-beamer:
\usepackage{cooltooltips}
\usepackage{colordef}
\usepackage{beamerdefs}
\usepackage{lvblisting}

% Change the pictures here:
% logobig and logosmall are the internal names for the pictures: do not modify them. 
% Pictures must be supplied as JPEG, PNG or, to be preferred, PDF
\pgfdeclareimage[height=2cm]{logobig}{hulogo}
% Supply the correct logo for your class and change the file name to "logo". The logo will appear in the lower
% right corner:
\pgfdeclareimage[height=0.7cm]{logosmall}{Figures/LOB_Logo}

% Title page outline:
% use this number to modify the scaling of the headline on title page
\renewcommand{\titlescale}{1.0}
% the title page has two columns, the following two values determine the percentage each one should get
\renewcommand{\titlescale}{1.0}
\renewcommand{\leftcol}{0.6}

% Define the title.Don't forget to insert an abbreviation instead 
% of "title for footer". It will appear in the lower left corner:
\title[Quiz 5]{Selected topics in Mathematical Statistics, Quiz 5}
% Define the authors:
\authora{Malte Esders} % a-c
\authorb{}
\authorc{}

% Define any internet addresses, if you want to display them on the title page:
\def\linka{http://lvb.wiwi.hu-berlin.de}
\def\linkb{}
\def\linkc{}
% Define the institute:
\institute{Humboldt--Universit�t zu Berlin \\}

% Comment the following command, if you don't want, that the pdf file starts in full screen mode:
%\hypersetup{pdfpagemode=FullScreen}

%Start of the document
\begin{document}

% create the title slide, layout controlled in beamerdefs.sty and the foregoing specifications
\frame[plain]{
\titlepage
}

%%%%%%%%%%%%%%%%%%%%%%%%%%%%%%%%%%%%%%%%%%%%%%%%%%%%%%%%%%%%%%%%%%%%%%%%%%%%%%%%%%%%%%%%%%%%%%%%%%%%%%%%%%%%%%%%%%%%%%%%
\section{Problem Description}

\frame{
\frametitle{Problem Description}
Quiz 5: Prove (18) under standard normal distribution, where (18) is:

If $\varphi_X$ is absolutely integrable,
\begin{equation*} 
	(18)\  \ f_X(x) = \frac{1}{(2\pi)}\ \int_{-\infty}^{\infty}{e^{-itx} \varphi_X(t)\ dt}
\end{equation*}					
}

\section{Lemma 1}

\frame{
\frametitle{Proof of Lemma 1}
To prove the statement, first we will need one Lemma:

\begin{equation}
	\text{Lemma 1:} \ \ \int_{-\infty}^{\infty}{e^{-\frac{1}{2}(x-it)^2}\ dx} = \sqrt{2\pi}
\end{equation}
}

\frame{
We start with
\begin{equation}
	\int_{-\infty}^{\infty}{e^{-\frac{1}{2}(x-it)^2}\ dx}
\end{equation}
\\

First, substitute $S = x-it$, we have

\begin{equation}
\begin{aligned}
	\int_{-\infty-it}^{\infty-it}{e^{-\frac{1}{2}S^2}\ dS}
\end{aligned}
\end{equation}
}

\frame{
We're first taking the integral between $\alpha$ and $-\alpha$ and later take the limits at $\infty$.
Consider the integral on a contour like this: $\mathcal{C} = \alpha \rightarrow -\alpha \rightarrow -\alpha-it \rightarrow \alpha-it \rightarrow \alpha.$ \\
Now since the normal distribution is analytic everywhere, we must have
\begin{equation}
	\oint_\mathcal{C}{f_X(z)\ dz} = 0
\end{equation}
}
\frame{
Writing out all four parts of the contour integral gives
\begin{equation}
\begin{aligned}
	\oint_\mathcal{C}{f(S)\ dS} &= \int_{\alpha}^{-\alpha}{e^{-\frac{S^2}{2}}\ dS} \\
	&+ \int_{-\alpha}^{-\alpha-it}{e^{-\frac{S^2}{2}}\ dS} \\
	&+ \int_{-\alpha-it}^{\alpha-it}{e^{-\frac{S^2}{2}}\ dS} \\
	&+ \int_{\alpha-it}^{\alpha}{e^{-\frac{S^2}{2}}\ dS} = 0
\end{aligned}
\end{equation}

}

\frame{
As we take the limits for $\alpha \rightarrow \infty$, the first term becomes $-\sqrt{2\pi}$ (because we're integrating from right to left), and terms 2 and 4 become zero. The third term is the term we're interested in. As we solve for that term, we get
\begin{equation}
	\int_{-\infty-it}^{\infty-it}{e^{-\frac{S^2}{2}}\ dS} = \sqrt{2\pi}
\end{equation}
which completes the proof.

\section{Proof of Statement (18)}
}
\frame{
\frametitle{Proof of Statement (18)}
In order to proof (18), we first compute the characteristic function $\varphi_X(t)$ of the standard normal distribution. From (17) we have
\begin{equation}
	\varphi_X(t) = \int_{-\infty}^{\infty}{e^{itx} \frac{1}{\sqrt{2\pi}} e^{-\frac{x^2}{2}}\ dx}
\end{equation}
}

\frame{
Looking at the exponent of $e$, we complete the square in t
\begin{equation} \label{eq:square_complete}
	\begin{aligned}
	\varphi_X(t) &= \frac{1}{\sqrt{2\pi}} \int_{-\infty}^{\infty}{e^{-\frac{1}{2} (x^2 +2itx -t^2)}  e^{-\frac{1}{2}t^2}\ dx} \\
	&= \frac{1}{\sqrt{2\pi}} e^{-\frac{1}{2}t^2}\ \int_{-\infty}^{\infty}{e^{-\frac{1}{2} (x^2 +2itx -t^2)}} \\
	&= \frac{1}{\sqrt{2\pi}} e^{-\frac{1}{2}t^2}\ \int_{-\infty}^{\infty}{e^{-\frac{1}{2} (x - it)^2}}
	\end{aligned}
\end{equation}
}

\frame{
	Now by Lemma 1, the Integral in the last line of (\ref{eq:square_complete}) is $\sqrt{2\pi}$, and $\varphi_X(t)$ is
\begin{equation}
	\varphi_X(t) = e^{-\frac{t^2}{2}}
\end{equation}
\\
To complete the proof we substitute $\varphi_X(t)$ into (18):
\begin{equation}
	f_X(x) = \frac{1}{2\pi}\ \int_{-\infty}^{\infty}{e^{-itx} e^{-\frac{t^2}{2}}\ dt}
\end{equation}
}

\frame{
We proceed by completing the squre similarly to (\ref{eq:square_complete}) and get
\begin{equation}
	f_X(x) = \frac{1}{2\pi} e^{-\frac{x^2}{2}}\ \int_{-\infty}^{\infty}{e^{-\frac{1}{2} (t - ix)^2}\ dt}
\end{equation}
Again by Lemma 1, the integral is $\sqrt{2\pi}$ and we are left with
\begin{equation}
	f_X(x) = \frac{1}{\sqrt{2\pi}} e^{-\frac{x^2}{2}}\ 
\end{equation}
which completes the proof.
}
%%%%%%%%%%%%%%%%%%%%%%%%%%%%%%%%%%%%%%%%%%%%%%%%%%%%%%%%%%%%%%%%%%%%%%%%%%%%%%%%%%%%%%%%%%%%%%%%%%%%%%%%%%%%%%%%%%%%%%%%
\end{document}
